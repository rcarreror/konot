\section{Fundamentos teóricos}

\subsection{Definición de latencia}
\lipsum[1-1]

\subsection{Factores que influyen en la latencia}
(a) Transmission time (serialization/encoding)
(b) Propagation time
(c) Queueing delay (buffering delay)
(d) Processing time
(e) Access to shared medium, like Wi-Fi channel acquisition delay, etc.
(f) Other overheads, like Ethernet preamble, etc.
\lipsum[1-1]

\subsection{Métricas}
\lipsum[1-1]

\subsection{Métodos de medición de latencia}
\lipsum[1-1]

\subsection{Calidad de servicio (QoS)}
Funcionamiento de colas en routers. Punto 3 paper "Model-based reinforcement learning for..." 

\lipsum[1-1]

\subsection{Calidad de experiencia (QoE)}
\subsubsection{Definición}
\subsubsection{Diferencia con QoS}
\subsubsection{Cómo afecta la latencia a los usuarios}
Comparar las distintas medidas con como afectan al usuario final.
Por ejemplo packet loss con cortes en vídeollamadas.

\lipsum[1-1]

\subsection{Tecnologías utilizadas por los usuarios}
\lipsum[1-1]